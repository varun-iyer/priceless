\documentclass[12pt]{article}
 
 
\usepackage{setspace}
\doublespacing
\usepackage{mathpazo}
\usepackage[margin=1in]{geometry}
\usepackage[backend=bibtex]{biblatex-chicago}
\bibliography{note}
\title{WRIT 105CD Multimodal Presentation Author’s Note}
\author{\emph{Varun Iyer}, \emph{Rukmini Bapat}, \emph{Sriya Aluru}, \emph{Ian Campbell}}
\date{\emph{\today}}
\begin{document}
	\maketitle	
	\section*{Introduction}
		Our game, \emph{Priceless}, is a city-builder game that explores the
		relationship between environmental policy and economic benefit. This
		author’s note elaborates on the issue we engage with and the narrative
		choices that advance our rhetoric regarding the issue.

	\section{Exigence}
		
		Environmentally-conscious decision making is important, complex, and
		value-laden. The ongoing climate crisis is just about as exigent as
		circumstances can be, and yet, responding to the crisis is incredibly
		difficult.

		As some other groups noted, many decision-makers do not even publicly
		believe in climate change, and others still are happy to be complacent
		to patently wrongful actions. Even among those that are
		well-intentioned and relatively like-minded, there is little consensus
		about how to respond to difficult environmental questions.  It
		is hard to define responsibilities and resolve trade-offs in this huge
		issue that touches nearly every person and aspect of life.  The experts
		of different disciplines approach environmental issues while holding
		different values, and make radically different recommendations as a
		result.

		Two works: \emph{Balancing on a Planet}, by Prof. David Cleveland (a
		research professor emeritus at UCSB) and \emph{Priceless: On Knowing
		the Price of Everything and the Value of Nothing}, by Frank Ackerman
		and Lisa Heinzerling, explore environmental issues, ways to address
		them, and the values we should hold when responding to climate change.
		Specifically, these writers criticize the discipline of economics for
		approaching environmental issues with a set of values and tools that are
		inappropriate to deal with values that are beyond price.

		Prof. Cleveland describes the fact-value distinction as “...the ability
		to distinguish between how the world is and how we would like it to
		be.”\autocite[5]{bop} According to Cleveland’s analysis of
		environmental literature in economics, sociology, and the environmental
		sciences, “we are usually unaware of how our value assumptions about
		how the world should work influence our empirical assumptions about how
		the world actually does work.”\autocite[5]{bop} Cleveland argues that
		we should examine our values and seek to keep them apart from empirical
		questions of science.

		On Cleveland’s view, economic analysis makes several value-lade
		assumptions that combine facts about the world (that can be proven) and
		values (that are ultimately revisable and debatable). Specifically, he
		finds that economists want to collapse all values onto a common scale:
		monetary price. This measurement best lends itself to the kind of
		cost/benefit analysis that economists can perform. Economists also
		assume that most resources are replaceable, and that new technology can
		always improve efficiency and resource output. Together, these
		assumptions translate into policy that is focused on producing more,
		but doing so ‘sustainably,’ using less inputs. At its core, the
		economic analysis does not question the core goal of maximizing
		material wealth.\autocite[105]{bop}

		Ackerman and Heinzerling concur with Cleveland: 
		\begin{quote} “The basic problem with narrow economic analysis of
		health and environmental protection is that human life, health, and
	nature cannot be described meaningfully in monetary terms; they are
priceless.”\autocite{pl} \end{quote} They both go further, arguing that
economists use cost-benefit analysis to cloak their work in a veneer of
objectivity.  “By proceeding as if its assumptions are scientific and by
speaking a language all its own, economic analysis too easily conceals the
basic human questions that lie at its heart and excludes the voices of people
		untrained in the field.”\autocite[9]{pl}

		Policy decisions (often made by economists) and the general public
		zeitgeist continue to embrace the economic mindset of continually expanding
		material wealth while attempting to reduce some of the environmental
		consequences on the back end. 

		To be clear here, both works recognize that many people continue to
		suffer through poverty and a lack of social and material resources. The
		aim of these works is not to advocate for all pursuit of material
		value, like resources that would help people in need. Rather, a
		part of the aim of these works is to stimulate reflection on why
		material value is valuable (exactly because it can alleviate suffering)
		and to remember that is a mere means to more important ends rather than
		an end in itself. This stands in contrast with the economics perspective
		which seeks to maximize material value alone without always connecting
		it to its underlying moral purposes.

		While Cleveland and Ackerman \& Heinzerling focus on different empirical
		issues, one of the themes running through both works is that we cannot
		consume or optimize our way out of environmental harms. Preserving that
		which we truly value may mean sacrificing measurable, material goods.

	\section{Artifact}

		Our artifact seeks to explore this theme --- the sacrifice of
		measurable, material goods in favor of securing true value.
		In John Ferrara’s “Games for Persuasion,” Ferrara articulates how games
		can be used as an effective persuasive tool. “If you make the core
		message into the secret of winning, then you will drive people
		efficiently toward that conclusion.”\autocite[299]{ferrara}

		This creates an interesting little paradox when applied to the exigence
		we have chosen. The primary themes that we draw upon from Cleveland and
		Heinzerling \& Ackerman is that environmental values cannot and should
		not be priced. Creating a game which rewards the player by a certain
		amount for ‘protecting the environment’ undermines the very message
		that we seek to communicate. The difficulty with making good
		environmental policy is that the environment (and other moral values,
		like human lives) are unmeasurable, invaluable goods that cannot be
		traded off or exchanged. Protecting the environment is unrewarding work
		that is more likely to entail sacrifice than distinction.

		Instead, our game encodes its message of reflection on environmental
		values into the aim of the game itself. We do employ Ferrara’s
		recommendation that game elements that are intended to be communicative
		about real-life matters should bear sufficient resemblance to the
		real-life matters so that the knowledge can transfer from the game into
		life. To this end, as in real life, economic value is represented by
		an easily measurable apparent goal. Increasing a player’s ‘city level’
		nets them apparent rewards, an expanded city, and more exciting art.

		As in the world, acquiring great material benefit requires
		damage to the environment surrounding the player. This damage is
		visually represented by the change of the tiles from verdant forest,
		mountain, and lake textures to barren, dry and depleted land. As in the
		world, there is no apparent, measurable benefit gained by preserving
		these resources.  It continues to be the case that it is
		simply the right thing to do.

	\section{Conclusion and Future Work}

		We hope that the design of the game and the map’s visual transformation
		from lush to lacking will cause the player to question their
		conceptions about the true aim of the game, and whether the development
		of their city was ‘worth it.’

		The visual art of this game forms an important part of its
		communicative element. The persuasive impact will be much greater if
		the game’s starting condition is truly visually appealing and rewards
		the user merely for being still and appreciating it --- for example,
		hearing a bird chirp if one watches a tree for a few minutes.

		The current art likely fails to create the necessary aesthetic and
		emotional connection for the game to carry out its intended persuasive
		effect to the extent possible.
	 

\end{document}
